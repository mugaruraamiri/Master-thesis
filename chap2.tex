\chapter{Related Works}
\section{ Mobile care (Moca) for Remote diagnosis and Screening}
Moca is a mobile application project that was developed by student at MIT. the system aims at providing diagnosis and screening services in resource poor settings and has been open sourced. with the growing number of mobile coverage in the world, the system uses mobile technologies that allows it to transmit different kind of data format such as: photos, x-rays images, audio files and text to the remote medical expert around the world. medical expert can design the work flow (Question-Answer pair) document and upload it to the mobile app where community health care workers can use to diagnose patients in remote areas. the work flow allows doctors to design effective decisions supports for the community health workers in the field. The system is designed to easily integrated with already existing health care systems such as OpenMRS or OpenROSA for portability and easy access\cite{celi2009mobile}.

\section{Health Care Based on IoT Using Raspberry Pi }

To get the information on Non communicable and communicable diseases such as  cardiovascular , diabetes and pneumonia is a challenge to many developing countries. Therefore , to overcome these problems some researches  started developing low cost medical devices which can be used to assist doctors and health workers to diagnose  or get an alert message in case of the emergency . As per (Surya et al 2015), it is demonstrated that by combining raspberry pi , GSM modem and ECG machine doctors and nurses will be able to get  the  heart beat data from patients located in different room of the hospital  in real time through the SMS alert , and web interfaces. However, this system is only designed to provide information of the patients who are already in the hospitals and developing countries like Rwanda need  a  low cost  mobile device that would enables the villages health workers to  follow up the patients who are in the  remote areas far from the hospitals  as well as providing information to the district doctors in real time\cite{msuryadeekshithguptavamsikrishnapatchavavirginiamenezes2015}. 




