\chapter{Introduction}

\section{Background}

Rwanda healthcare have shown growth for the last two decades. In fact it is now the only subsaharan country in the way to achieve the millennium development goals in healthcare\cite{farmer2013reduced}. Rwanda now registers close to 90\% 90 per cent coverage of health insurances one of the best in the entire continent\cite{twahirwa2008sharing}. A lot of  reforms have been put in place after 1994, and implementing different programs to support the health care systems. For instance, one of the big and most effective program in place is the community health workers. this program has helped in reducing the number of children death and maternal mortality death Through their training on providing basic health care services to patient in their homes\cite{theunitednationschildrensemergencyfundunicef2013}.The biggest cause of the deaths including: malaria, diarrhea, respiratory diseases such as pneumonia and HIV/ADIS just to name a few\cite{chen2004human}. The CHWs program assigns each cells with three CHW, where two in the group named Binome are in charge of basic health care need and children care between the age of 6 and up. the remaining one is in charge of women and newborn care. CHWs has proven to be effective, where it has alleviated the issue of shortage of human resources especially in rural community\cite{condo2014rwanda}. It is effective not only by providing basic healthcare to the community but also being a liaison between the citizens to the formal healthcare system.

Despite all the achievement Rwanda has made in the past, but like any other developing country, Rwanda medical needs still outgrew its medical resources. there are still issues that hinders the delivery of healthcare services to all the citizens of the country. those issues includes:

\begin{itemize}
	\item Limited number of qualified human resource such as Doctors and nurses. In Rwanda there is only one doctor per 15428 inhabitants and one nurse per 1200 inhabitants (NISR Project 2012 census). There is a huge gap in Doctor-to-patient and Nurse-to-patient ration and this result into long queues of patients to clinics and hospitals waiting to get served. Patient might spend a lot of hours and may be days without gating access to a doctor when they are already at the clinic
	\item Lack of proper and sufficient equipment and tools. Many private health institutions as well as small public health care center in rural areas not only have limitation in numbers of skilled personnel but also limited capability of providing quality health services, and this many because they lack many equipment and tools that help them to do so. this issue also affect the perforce of community health workers, since most of the do not have the kind of tools when performing their diagnosis. for instance, they have to use a watch timer when diagnosing respiratory diseases when counting the number of time a patient perform the inhalation and exhalation of the air in the lungs per minute. Many of the method to diagnose their patients are base to just observing by their eyes and make assumptions basing on the sings observed. Although this might works in some cases, but can also result into making mistakes that can put the patient into danger
	\item Limited access to information is another great challenge faced by Healthcare systems in Rwanda. The current means of information sharing are between community health workers, clinics and the ministry of Health and they are mostly for report submissions via SMS based platform such as Rapid SMS and Web based platform such as OpenMRS. Currently there are no system in place that can share patient diagnosis information to Doctors or other stakeholders in real time or system that can deliver timely feedback and small Interventions to assist healthcare workers remotely and help them make better decisions.
	\item In addition to the above challenges, Geographic barriers are also a problem to the community especially those in remote areas. Rwanda is a mountainous country with few basic infrastructures like roads, bridges and other facilities that can allow patient to get quickly to the nearest hospitals or clinics. As per Ulises and Carina analysis, only 26\% of the population in Rwanda are within the reach of health care facility by walk, and most of the population in the wester province can not easily access to basic health care needs\cite{munoz2012geographical}.According to Household HealthCare Access survey conducted by IMS Consulting Group in 2012, it is say that people in resource poor settings areas in a country like India 32 per-cent of people in those areas travels more than 5km to seek OPD (Outpatient Department) treatment
\end{itemize}

Just to mention a few, these are the main challenges that the healthcare sector in Rwanda faces. and it is more challenging in rural areas due to their poor settings and many other related issues that were mentioned above. The mentioned problems present a big scope that we can not cover in this work due to the short amount of time. In this work we will only focus on how we can effectively use open-source technology to assist CHWs around the country in Rwanda. We want to have a system that can help detect and prevent diseases at their early stages by providing that first step basic patient diagnosis.

\section{Objective}
The main objective of this works is to try and provide the affordable and easy way to the Community Health workers to provide basic and accurate yet being effective diagnosing tools using open-source technologies in rural community.

also to allow them have access to quick information that can make decision making quick and more effective. Learning on the go is one of the most important part of the process that can help in saving more lives and avoid making mistakes which is also this work will be aiming at.

to try and bring together those areas with poor setting to those with more advanced settings such as the big hospitals in the cities. this work will allow CHWs work with Doctors in urban areas in real time. Doctors to perform remote diagnosis and share feedbacks with CHWs to assist them when on the field.

\section{Structure of the paper}


